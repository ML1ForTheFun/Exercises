\documentclass[11pt,a4paper]{article}
\usepackage{amsmath}
\usepackage{amssymb}
\usepackage{graphicx}
\usepackage{verbatim}
\begin{document}
\noindent
Martin Lundfall, Henri Bunting, Malte Siemers, Patrik Bey
\begin{centering}
  \section*{Exercise sheet 10 - Machine Intelligence I}
  \end{centering}
\section*{10.3}
\subsection*{a)}
Representing the probabilities as a DAG, we see the following dependencies:
\begin{figure}[h]
  \centering
  \includegraphics[width=.4\textwidth]{graph}
    \caption{DAG illustrating probability dependencies}
  \end{figure}
\subsection*{b)}
Explaining away the probabilities given, we consider the case where you're alarm has gone off while there was a radio broadcast. What are the respective probabilities of a burglary having occured?\\
The probability of a burglary not having occured is:
\begin{equation*}
  \begin{split}
  P(B=f|A=t,E=t)  = \frac{P(E=t, A=t | B=f)P(B=f)}{P(A=t,E=t)}=\\
  \\
  =  \frac{P(A=t |E=t, B=f)P(B=f)P(E=t)}{P(A=t|E=t)P(E=t)}= \\
  \\
  =  \frac{P(A=t| B=f, E=t)P(B=f)}{P(A=t|B=t,E=t)P(B=t)+P(A=t|B=f,E=t)P(B=f)} = \frac{0.41*0.99}{0.98*0.01+0.41+0.99}
  \end{split}
\end{equation*}
The probability of a burglary having occured is:
\begin{equation*}
  \begin{split}
  P(B=t|A=t,E=t)  = \frac{P(E=t, A=t | B=t)P(B=t)}{P(A=t,E=t)}= \\
  \\
  =  \frac{P(A=t |E=t, B=t)P(B=t)P(E=t)}{P(A=t|E=t)P(E=t)}= \\
  \\
  =  \frac{P(A=t| B=t, E=t)P(B=t)}{P(A=t|B=t,E=t)P(B=t)+P(A=t|B=f,E=t)P(B=f)} \\
  \frac{0.98*0.01}{0.98*0.01+0.41*0.99}=0.02357...
  \end{split}
\end{equation*}
We see that if there was a radio broadcast while the alarm went off, a burglary probably hasn't occured. This seems to indicate that keeping the radio on reduces the risk for burglaries, but this is a faulty assumtion to make, since the two probabilities are independent.
\end{document}
