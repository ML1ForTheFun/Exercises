\documentclass[11pt,a4paper]{article}
\usepackage{amsmath}
\usepackage{graphicx}
\usepackage{verbatim}
\begin{document}
\section*{2.1}
\subsection*{a}
A nonlinear transfer function gives the neural network a universal property: Given enough layers and neurons, the network can model any function within a certain accuracy. In a network with a linear transfer function we can only compute a linear function. A network with a linear transfer function of $n$ layers will always be equivalent to a network with only one layer: linear functions can always be concatinated.\\
Whenever the function that we are trying to model isn't a linear function, it is useful to use a nonlinear transfer function.
Examples include image classification or speech recognition.
\subsection*{b}
Consider a simple neural network with two input neurons that can both either be 0 or 1 and one output layer. We want to construct an AND gate with our network, so without bias our quest would be to find $w = (w_1, w_2)$ such that:
\begin{align*}
  0 \leq 0\\
  w_1 \leq 0\\
  w_2 \leq 0\\
  w_1 + w_2 > 0\\
\end{align*}
which is impossible. We can easily however create the network with a bias, if we have the weights $w = (1, 1)$ and the bias
$\theta = \frac{3}{2}$. Then $sgn(w^Tx - \theta)$ would give us AND.
\subsection*{c}
Point and edge filters are for example a connectionist neuron which gets values of a scalar field as input, that
represent the color of each pixel or the color gradient or even a higher derivative and has weights in the following form:
\begin{figure}[h]
\centering
 \includegraphics[width=0.4\textwidth]{./point_edge.png}
\end{figure}
In the simple case of two colors (0 and 1) this point filter would return zero for no point, 1 or -1 for a point in the outer
region and 8 or -8 if the point is in the middle. This goes analogously for the other filter.
\subsection*{d}
The first is deterministic and the second has a noise parameter and can return different states for set parameters and a given
input.
\section*{2.2}
The code for this assignment can be found in the file \verb|2_2.py|
\subsection*{a}
\begin{figure}[h]
  \centering
  \includegraphics[width=\textwidth]{Input data with classification.png}
  \caption{Input data, x[1] against x[0] colored according to y value}
\end{figure}
\subsection*{b}
\begin{figure}[h]
  \centering
  \includegraphics[width=\textwidth]{Classification performance for various w-angles.png}
  \caption{Classification performance given by $sign(w \cdot x)$ for varying angles determining w}
\end{figure}
\newpage
\subsection*{c}
The weight vector giving us the best performance is $w=(0.93969262,  0.34202014)$ if we optimize it without respect to $\theta$.
\subsection*{d}
\begin{figure}[h]
  \begin{centering}
  \includegraphics[width=\textwidth]{Classification performance with w and theta optimized separately.png}
  \caption{Classification performance with $w$ and $theta$ optimized separately}
  \end{centering}
The weight vector $w$ is colored pink in the figure, and it is perpendiculair to the line along which a lot of x are classified. $w*x$ gives us a measure of how far away our points, $x$ are from this line.
If we optimize $w$ and $\theta$ one after another we get a  $p = 0.805$  classification rate and an optimal $\theta = 0.135135135135$.
\end{figure}
\newpage
\subsection*{e}
If we optimize w and theta one after another we get a  $p = 0.915$ classification rate with optimal parameters $w = (0.64278761,  0.76604444)$ and $\theta = 0.339339339339$
\newpage
\section*{2.3}
\subsection*{a}
A MLP could decide between a horizontal and a vertical edge, whereas a perceptionist neuron would either be able to differ between
vertical edge or no vertical edge OR horizontal edge or no horizontal edge.
\subsection*{b}
\begin{figure}[h]
\centering
 \includegraphics[width=\textwidth]{./2_3_std=2.png}
\caption{functions computed with normally distributed $a_i$ with a standard deviation of 2}
\end{figure}
\newpage
\subsection*{c}
\begin{figure}[h]
\centering
 \includegraphics[width=\textwidth]{./2_3_std=2.png}
\caption{functions computed with normally distributed $a_i$ with a standard deviation of 2}
\end{figure}
\subsection*{c + bonus}
\begin{figure}[h]
\centering
 \includegraphics[width=0.75\textwidth]{./2_3_std=0_5.png}
\caption{functions computed with normally distributed $a_i$ with a standard deviation of 0.5}
\end{figure}
\begin{figure}[!h]
\centering
 \includegraphics[width=0.75\textwidth]{./2_3_star.png}
\caption{functions with least mean square error from f(x)=-x, computed with an std of 2 (green) and 0.5 (blue)}
\end{figure}
\end{document}
