\documentclass[11pt,a4paper]{article}
\usepackage{amsmath}
\usepackage{amssymb}
\usepackage{graphicx}
\usepackage{verbatim}
\begin{document}
\noindent
Martin Lundfall, Henri Bunting, Malte Siemers
\begin{centering}
  \section*{Exercise sheet 06 - Machine Intelligence I}
  \end{centering}
\section*{6.4}
Since the logarithm is a strictly convex function (at least for positive arguments), then \textit{Jensen's inequality} implies:
\begin{equation}
  D_{KL}(p||q):= \sum_{x \in X}p(x)log\frac{p(x)}{q(x)}  \geq log \sum_{x \in X}(p(x)\frac{p(x)}{q(x)}) = log(1) = 0
\end{equation}
If $\forall x: p(x)=q(x)$ this becomes:
\begin{equation}
D_{KL}(p||q):= \sum_{x \in X}p(x)log\frac{p(x)}{q(x)} = \sum_{x \in X}p(x)log(1) = 0
\end{equation}
If $D_{KL}(p||q) = 0$, then we get:
\begin{equation}
  \sum_{x \in X}p(x)log\frac{p(x)}{q(x)}=0 \implies \sum_{x \in X}(p(x)) \sum_{x \in X}log\frac{p(x)}{q(x)} = 0
\end{equation}
Since $p(x)$ and $q(x)$ are always positive, this implies that for all $x \in X$ we must have:
\begin{equation*}
  log\frac{p(x)}{q(x)} = 0 \implies \frac{p(x)}{q(x)} = 1 \implies p(x)=q(x)
\end{equation*}
\end{document}

